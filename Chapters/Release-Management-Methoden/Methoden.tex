\chapter{Release Management Methoden}

\section{Ad-hoc Methodik}

Dieser Ansatz konzentriert sich auf menschliche Intuition, Kommunikation und menschlichen Fähigkeiten, um zu entscheiden, welche Anforderungen für welche Releases ausgewählt werden. Ein Ad-hoc-Ansatz kann für relativ kleine interne Projekte mit wenigen Dutzend Anforderungen und lockeren Einschränkungen geeignet sein \cite{Mohamed2016:Devops}.

\section{Inkrementelle Methode}

Beim inkrementellen Ansatz werden Anforderungen in mehrere unabhängige Module des Softwareentwicklungszyklus unterteilt. Die inkrementelle Entwicklung erfolgt in folgende Schritten:
\begin{itemize}
	\item Analyse und Design
	\item Implementierung
	\item Testen/Verifizieren
	\item Wartung
\end{itemize}

Hier wird das ganze Projekt in kleinere Miniprojekt unterteilt. Die Vorteile davon ist, dass die Software können während des Softwarelebenszyklus generiert. Dies ermöglicht es Endbenutzern, die Komponenten des Systems im Voraus zu erhalten,
was ihren Geschäftswert erhöht und eine zeitnahe Reaktion gewährleistet. Zudem sind sich ändernde Anforderungen und Untersuchungen flexibel und kostengünstig. Außer dem Fehler sind leicht zu erkennen.Diese Methode erfordert jedoch eine
gute Planung, um Fehler oder Probleme zu beheben, und sie wird in allen Blöcken benötigt, was viel Zeit verschwendet.

\section{Agile Methodology}

In der agilen Softwareentwicklung werden Anforderungen und Lösungen durch die
Zusammenarbeit selbstorganisierender, funktionsübergreifender Teams entwickelt.
Der Zweck dieses Ansatzes ist es, eine motivierte, organisierte und verantwortungsvolle
Teamarbeit aufzubauen, um eine schnelle Lieferung qualitativ hochwertiger
Software zu gewährleisten\cite{digite.com:agile}. Darüber hinaus ermutigen agile Ansätze Softwareentwickler,
ihre Softwareanforderungen in kleinere Releases, sogenannte User Stories,
zu unterteilen, um Kundenfeedback und -reaktionen zu beschleunigen.
Agile ist ein Überbegriff für mehrere Methoden und einige der beliebtesten Methoden
sind im Folgenden aufgelistet:

\begin{itemize}
	\item Scrum
	\item Extreme Programming
	\item Adaptive Software Development
	\item Dynamic Software Development Method
	\item ...
\end{itemize}

Mohamed \cite{FernandoBritoE:AbreuDaSilva} evaluierte, dass die Agile Ansätze sei Problematisch, wenn es zu einer Interaktion mit anderen Unternehmen
kämme, die unterschiedliche Kulturen, Arbeitsansatz, Arbeitsumfang und Geschäftsprozesse
haben.

\section{DevOps Methodologie}

Diese Software Entwicklungsmethodik bringt Entwicklerteam und Betriebsteam zusammen.
Traditionell haben die beiden Teams getrennt gearbeitet, von der Entwicklung
bis zur Veröffentlichung von Software oder Firmware. Dies dient der Kommunikation
zwischen ihnen, damit sie Software schneller und effizienter erstellen, testen
und veröffentlichen können\cite{Novatec}.

\section{Continuous Delivery Methodologie}

\ac{CD} basiert auf dem Agile Manifesto-Prinzip: "Höchste Priorität ist es, den Kunden mit der zeitnahen und kontinuierlichen Lieferung wertvoller Software zu erfreuen". Der Zweck der \acs{CD} besteht darin, den Software Lieferprozess durch eine weitgehend automatisierte Pipeline zu verbessern, um zuverlässige Software-Releases in einem kurzen Zeitrahmen zu gewährleisten.
