\chapter{Zusammenfassung}

In dieser Arbeit wurden vorhandene Methoden untersucht,um die Software der verteilten
simulierten Ampelanlage zu aktualisieren. Aufgrund der enormen Vorteile von SOTA wurde es in diesem Projekt als Software-Update-Methode verwendet. Für die Umsetzung dieser Methode wurde eine experimentelle Umgebung mithilfe der Raspberry Pi und die GPIO Schnittstelle aufgebaut. Darüber hinaus wurde der Release-Management-Prozess auf ein effizientes Software-Update überprüft. Für die Planungs- und Entwicklungsphase wurde ein Konzept entwickelt, das eine strukturierte und organisierte Entwicklung erreicht. Dieses Konzept wurde mithilfe von VCS GitLab ermöglicht und bietet auch ein übersichtliches Versionierungsverfahren für die Release-Phase einer Software. Zudem wurde die CD-Methode gewählt, um das Software-Update kostengünstig und schnell ans Ziel zu bringen. Diese Methode automatisiert die drei Phasen des Release-Management-Prozesses (Build, Release und Deployment).Bei der CD-Methode wurde für jeden Schritt eine geeignete Lösung erarbeitet und durch den geeigneten Runner ausgeführt.
\newline \newline
Abschließend lässt sich sagen, dass alle zusammenhängenden CI/CD-Praktiken zusammen das Risiko der Bereitstellung einer Anwendung reduzieren, da es einfacher ist, Änderungen an einer Anwendung in kleinen Teilen statt auf einmal freizugeben.
