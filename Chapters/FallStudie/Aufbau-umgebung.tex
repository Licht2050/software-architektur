\subsection{Aufbau der experimentellen Umgebung}

Zur Ausführung der vorher ausgewählten SOTA-Methode werden ein Server und ein Client benötigt. Der Server kommuniziert mit dem Client über die in das Produkt eingebaute Internetschnittstelle. Dadurch wird das Update übertragen und während der Übertragung Nachrichten ausgetauscht. Für diese Aktualisierungsmethode ist eine Testumgebung konfiguriert. Die Umgebung umfasst ein Server-zu-Client-Kommunikationssystem. Als Server und Clients werden die Raspberry Pis  zunutze genommen. Ampelanlagen sind mit den Raspberry Pis, die als Client vorgesehen sind, verbunden. In diesem Projekt werden die reale Ampelanlage durch LEDs und Breadboard simuliert, um die serielle Kommunikation zu verwirklichen.
\newline\newline
Außerdem wird die Kubernetes-Cluster konfiguriert, der Master-Node wird auf einem Raspberry pi installiert und die Worker-Nodes sind in dieser Architektur die Raspberry Pis, die jeweils mit einer Ampelanlage verbunden sind. Die installierte Worker-Nodes sollen mit dem Master-Node registriert werden. Um Kubernetes-Anwendungen einfach nutzen und verwalten zu können, wird auch Helm-Tool auf den Master-Node heruntergeladen.