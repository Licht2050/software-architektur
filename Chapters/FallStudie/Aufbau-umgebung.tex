\subsection{Aufbau der experimentellen Umgebung}

Zur Ausführung der vorher ausgewählten SOTA-Methode werden ein Server und ein Client benötigt. Der Server kommuniziert mit dem Client über die in das Produkt eingebaute Internetschnittstelle. Dadurch wird das Update übertragen und während der Übertragung Nachrichten ausgetauscht. Für diese Aktualisierungsmethode ist eine Testumgebung konfiguriert. Die Umgebung umfasst ein Server-zu-Client-Kommunikationssystem. Als Server und Clients werden die Raspberry Pis  zunutze genommen. Ampelanlagen sind mit den Raspberry Pis, die als Client vorgesehen sind, verbunden. In diesem Projekt werden die reale Ampelanlage durch LEDs und Breadboard simuliert, um die serielle Kommunikation zu verwirklichen.
