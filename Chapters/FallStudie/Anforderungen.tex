\section{Anforderungen}

In diesem Abschnitt werden die funktionale und nicht-funktionale Anforderungen für die Realisierung des Projekts analysiert.


\subsection{Funktionale Anforderungen}

Die funktionalen Anforderungen beschreiben die gewünschte Funktionalität und das Verhalten des Systems.


\begin{table}[h!]
	\begin{flushleft}
		{\small 
			\begin{tabular}{|c|c|}
				\hline
				Must-have & 
				\begin{minipage}{5in}
					 \begin{enumerate}
					 	\item Simulierung zwei Verkehrsampeln mithilfe von Raspberry Pi’s und LEDs
					 	\item Automatische Steuerung der LEDs durch GPIO Schnittstelle
					 	\item Entwicklung einer Software für das Blinken der gelben LED
					 	\item Entwicklung einer Software für den Normalbetrieb der Ampel
					 	\item Automatische Software-Test
					 	\item Automatische Bereitstellung (release) der Software
					 	\item Implementierung von CI/CD Software-Entwicklungsmethode
					 \end{enumerate}
				\end{minipage} \\
				\hline
				Should-have & 
				\begin{minipage}{5in}
					\begin{enumerate}
						\item Automatische Containerisierung der für das Ampelsystem entwickelten Software mithilfe von Docker
						\item Automatische Veröffentlichung von containerisierten Software auf Docker Hub
						\item Orchestrierung der Docker-Containers mithilfe von Kubernetes
					\end{enumerate}
				\end{minipage} \\
				\hline
				Could-have & 
				\begin{minipage}{5in}
					\begin{enumerate}
						\item Automatische Übertragung (deploy) der Software an das Endgerät
					\end{enumerate}
				\end{minipage} \\
			\hline
			\end{tabular}	
		}
	\end{flushleft}
	\caption{Funktionale Anforderungen}	
\end{table}

\subsection{Nicht-Funktionale Anforderungen}