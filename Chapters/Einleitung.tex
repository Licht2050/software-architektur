\chapter{Einleitung}

\subsection{Motivation}

Die Notwendigkeit von Software-Updates ist sehr wichtig, um die korrekte Funktion
der eingebetteten Systeme fehlerfrei und problemlos zu gewährleisten, denn die Software-Bugs hatten katastrophale Folgen, insbesondere in den Bereichen Luft- und Raumfahrt, Automotive und Medizin. Zwischen Juni 1985 und Januar 1987 verabreichte ein computergesteuertes Therapiegerät namens “Therach-25” sechs Personen eine große Anzahl von Überdosierungen, die zu schweren Verletzungen und zum Tod führten.
Der Software-Entwicklern sollten Funktionen zur Verfügung gestellt werden, die den Software-Entwicklungsprozess beschleunigen und Software-Updates über das Internet ermöglichen. Denn ein manuelles Update, zum Beispiel über USB-Sticks, zum einen verlangsamt der Übertragungsprozess und zum anderen kann während der Übertragung zu menschlichen Fehlern führen. Darüber hinaus ist dieser Prozess anfällig, wie Fiat Chrysler 2015 damit begonnen hat, einen Hotfix für Millionen von Fahrzeugen per Post über einen USB-Stick zu verteilen, der es Hackern ermöglicht, Briefe und USB-Sticks zu fälschen.


\subsection{Aufgabenstellung und Zielsetzung}