\chapter{Einleitung}

\section{Motivation}

Die Notwendigkeit von Software-Updates ist sehr wichtig, um die korrekte Funktion
der eingebetteten Systeme fehlerfrei und problemlos zu gewährleisten, denn die Software-Bugs hatten katastrophale Folgen, insbesondere in den Bereichen Luft- und Raumfahrt, Automotive und Medizin. Zwischen Juni 1985 und Januar 1987 verabreichte ein computergesteuertes Therapiegerät namens “Therach-25” sechs Personen eine große Anzahl von Überdosierungen, die zu schweren Verletzungen und zum Tod führten.
Der Software-Entwicklern sollten Funktionen zur Verfügung gestellt werden, die den Software-Entwicklungsprozess beschleunigen und Software-Updates über das Internet ermöglichen. Denn ein manuelles Update, zum Beispiel über USB-Sticks, zum einen verlangsamt der Übertragungsprozess und zum anderen kann während der Übertragung zu menschlichen Fehlern führen. Darüber hinaus ist dieser Prozess anfällig, wie Fiat Chrysler 2015 damit begonnen hat, einen Hotfix für Millionen von Fahrzeugen per Post über einen USB-Stick zu verteilen, der es Hackern ermöglicht, Briefe und USB-Sticks zu fälschen.


\section{Aufgabenstellung und Zielsetzung}

In dieser Arbeit soll die verschiedenen Release-Management-Methoden erklärt und insbesondere die Methode Continuous Integration, Delivery and Deployment (CICD) herausgearbeitet und die verschiedenen dabei eingesetzten Tools sowie deren Unterschiede erläutert werden. Als nächstes wird auf die Frage eingegangen, welche CI/CD-Auswirkungen die Softwarearchitektur und die Entwicklungsprozesse von haben. Darüber hinaus wird die Anwendung dieser Methode anhand eines Fallbeispiels näher untersucht, in dem der Fall der Anwendung eines Software-Updates eines Endgeräts über das Internet tatsächlich betrachtet und unter Verwendung einer Container-Technologie namens Docker umgesetzt wird.

\section{Überblick}

Die vorliegende Ausarbeitung besteht aus 7 Kapiteln. In Kapitel 2 werden die wichtigsten Begriffe und Konzepte erläutert und in sieben Abschnitte unterteilt, die jeweils einen Teil des zum Verständnis dieser Arbeit erforderlichen Grundwissens vermitteln. Das Kapitel beginnt mit der Definition von Container- und Docker-Technologie. Außerdem wird Kubernetes und dessen Aufbau näher betrachtet. Dann wird der Einplatin Computer bzw. Raspberry Pi definiert. Ebenso wird der Konzept des Versionsverwaltungssystems kurz erläutert. Abschließend wird über das Release Management und auch die Typologien der Softwareverteilung erklärt.
\newline\newline
Kapitel 3 beschreibt verschiedene Release Management Methoden nämlich die Ad-hoc-, Inkrementelle-, Agile-, DevOps- und Continuous Delivery-Methoden.
\newline\newline
Kapitel 4 geht um die \ac{CI/CD}. Dort wird sowohl in der Definition als auch in dem Konzept von \ac{CI}, Continuous Delivery und Continuous Deployment (CD) vertieft. Außerdem, werden verschiedene CI/CD Tools aufgelistet und jeweils kurz darüber gesprochen.
\newline\newline
Die Fallstudie wird im Kapitel 5 vorgestellt. Hier wird die Anforderungsanalyse und die Durchführung der Anforderungsanalyse erforderliche Konzeption näher beschrieben und auf die Implementierung des Projekts \glqq Verkehrsampel\grqq eingegangen.
\newline\newline
Die Bewertung bzw. Evaluation der implementierten Lösung wird in Kapitel 6 beschrieben und schließlich die Zusammenfassung im Kapitel 7 enthalten.
