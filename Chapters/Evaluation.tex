\chapter{Evaluation}

In diesem Abschnitt werden die Konzeption und Implementierung zur Erfüllung der funktionalen und nicht-funktionalen Anforderungen bewertet. Das Ziel des Projekts ist die Simulation von zwei Verkehrsampeln, die mithilfe von Raspberry Pi’s, LEDs und CI/CD Software-Entwicklungsmethode realisiert werden. 
\newline\newline
Es wurde am Anfang zwei verschiedene Softwares für das Blinken vom gelben LED und normal Betrieb der Verkehrsampeln mithilfe von Python entwickelt. Die Software-Änderungen wurden kontinuierlich commited und in den remote branch gepusht. Zunächst wurden die entwickelte Softwares nach dem öffnen vom Pull-Request durch einen selbst geschriebenen Unit-Test für die Fehlerfreiheit des Softwares automatisch überprüft. Nach bestandener Unit-Test wurden die Software-Änderungen automatisch in den base Branch gemergt. Wenn die Software-Änderungen erfolgreich in den Produktionsumgebung bzw. \textit{main} branch gemergt sind, wird dann die Continuous Delivery/Deployment Phase gestartet. Das Entwicklerteam kann an dieser Stelle durch die Bearbeitung von \textit{.gitlab-ci.yml} Datei festlegen, wann jede Stage der Pipeline gestartet werden soll. In diesem Projekt wurde festgelegt, dass die Continuous Delivery/Deployment Stage nach dem erstellen von einem neuen Tag auf dem main branch gestartet wird. Zuerst wurde der Delivery Stage anfangen, in dem das Update automatisch in Form eines Docker-Images gekapselt und auf Docker Hub hochgeladen. Danach wird einen neuen Release des Softwares automatisch erstellt. Zunächst wird einen temporären Runner auf dem Master-Node registriert und einem Worker-Node zugewiesen. Am Ende wird der bereitgestellte Software auf dem entsprechenden Ziel-Worker-Node mithilfe von Kubernetes übertragen.
\newline\newline
Als Sicherheitsmechanismus wird die Pipeline die weitere Ausführung anderer Pipeline-Stages stoppen wenn etwas schief läuft. Die Abbildung~\ref{fig:GitLabCDPipeline} zeigt wie die Pipeline von Continuous Delivery und Deployment nach einer erfolgreichen Durchführung aller Pipeline-Stages aussieht.

\begin{figure}[!htbp]%[bth] 
	\centering
	\includegraphics[width=1.0\textwidth]{Graphics/GitLab-pipeline.png}
	\caption{GitLab CD Pipeline}
	\label{fig:GitLabCDPipeline}
\end{figure}
 
